\documentclass{article}
\usepackage[utf8]{inputenc}

%reliability factor, cost, outputs (soil/ph/etc)
%use research papers and articles to gather info

\title{MINTS Soil Sensors}
\author{Vardhan Agnihotri, Seth Lee}
\date{June 20 2022}

\begin{document}

\maketitle

\section{Introduction}
In this analysis, we present our findings on a multitude of soil sensors, all of which have the potential to be useful to the MINTS soil research program. Given the variations among these devices (in price, reliability, build quality, technical popularity, etc.), sensors belonging to a unique category will be collectively examined in accordance with several criteria which we determined to be pertinent to MINTS soil research:
\begin{enumerate}
    \item Reliability and cost effectiveness
    \item Outputs
    \item Quality %could be removed on the basis of the lack of info available to us`
    \item Popularity within the scientific community (based on scholarly articles and papers)
\end{enumerate}


\section{Sensor Profile 1: Probes}
The probes included within this section have the ability to measure both soil moisture and temperature

\subsection{Probes}
\begin{enumerate}
\item GroPoint Profile: Multi-Segment Soil Moisture and Temperature Profiling Probes
\begin{itemize}
    \item Utilizes Time Domain Transmissometry (TDT) to accurately measure soil's dielectric properties (which are governed by moisture content) over the entirety of the sensor 
    \item Provides cost effective soil moisture measurement over several depths using a single probe, eliminating the need for multiple cumbersome sensors and reducing the manufacturing cost
    \item Entirety of the sensor's circuit board (including the antenna) is housed within a polycarbonate housing sealed with epoxy, resulting in maximum durability
    \item Sensors are deployed in scientific research within the fields of farming, agriculture, mining, and hydrology (see attached bibliography)
    %cite later
\end{itemize}

\item Irrometer Tensiometers
\begin{itemize}
    \item First developed in 1951, remains the standard tensiometer (soil moisture measurement device) for scientific agricultural research to this day
    \item Generally inexpensive and reliable - comes in three different models that are all suited to their own respective environments
    \item Membrane-vented cover prohibits entry of dirt or other particulates into device, thus ensuring both accuracy and longevity of the product
    %cite articles
\end{itemize}

\end{enumerate}

\subsection{Conclusion}
Althugh the GroPoint Multi-Segment Probe proved to be useful in cutting costs, we claim that the Irrometer Tensiometer's superior reliability, popularity amongst researchers, and acceptable cost establish it as the better option for the MINTS soil research team.


\section{Sensor Profile 2: Shallow/Single Depth Sensors}
For the purposes of the MINTS soil research, shallow/single depth sensors are sensors that collect data on soil's moisture and temperature, similar to the probes above.  %In this section, we judge each sensor upon the factors laid out in the introduction above (examining each sensor's cost effectiveness, reliability, build quality, and outputs). %might remove last sentence

\subsection{General Single-Depth}
\begin{enumerate}
\item Milesight LoRaWAN Soil Moisture, Temperature, and Electrical Conductivity Sensor (EM500-SMTC)
\begin{itemize}
\item Capable of measuring and outputting soil moisture, temperature, and electrical soil conductivity (soil moisture and electrical soil are particularly useful in agricultural monitoring and examining the concentration of soil nutrients)
\end{itemize}
\end{enumerate}

\subsection{NPK}

\subsection{pH}

\subsection{Conclusion}

\section{Sensor Profile 3: Various Depth Sensors (Includes Multi-Depth)}
The sensors included within this section all have the ability to record data at different soil depths, whether discretely or simultaneously.

\subsection{Various Depth}  
\begin{enumerate}
\item Sensoterra Soil Moisture Sensors
\begin{itemize}
    \item Available in several depth sizes (15cm, 30cm, 60cm, 90cm); takes hourly measurements on soil moisture and uses the LoRaWAN network to transmit this data wirelessly
    \item Low-cost (pricetag of \$300) and accurate - proven by
    %https://www.mdpi.com/1424-8220/20/24/7041
    %price info given by https://dspace.mit.edu/handle/1721.1/129273
    \item Internal housing for onboard electronics enables an effective system that is largely unhindered by surroundings particulates
    \item Is popular among researches seeking low-cost, durable, and accurate moisture sensors
\end{itemize}
\end{enumerate}

\subsection{Multi-Depth}
\begin{enumerate}
\item Sensoterra Multi-Depth Sensors
\begin{itemize}
    \item Comes with 6 soil moisture sensors and a temperature sensor attached (meaning it can simultaneously track soil metrics at several different depths), uses LoRaWAN to relay data (akin to the variable-depth sensor)
    \item Costs \$500
    \item Similar design to the aforementioned single-depth version, which itself seemed robust and durable
    \item As a result of its heftier price, the sensor doesn't find much use within the scientific community
\end{itemize}
\end{enumerate}

\subsection{Conclusion}
We conclude here that to recommend one Sensoterra Variable-Depth sensor over the other is impractical, as both have their own situational strengths: the variable single-depth devices come at a low cost, offering less flexibility with measurement, while the multi-depth sensors come with additional moisture and temperature probes at a higher price.
%multi depth prob better

\end{document}
